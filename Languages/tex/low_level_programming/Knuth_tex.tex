\documentclass{article}
\usepackage[utf8]{inputenc}
\usepackage[T1]{fontenc}

% Set indentation to-------------------------------------------------------------------------------- zero
\setlength{\parindent}{0ex}

\begin{document}

\section{101}

% count<0~256> ->  32bits int
	\count2=42
	The value is now ‘\the\count2’ \\
	\def\macro{-1234}
	\count0=\macro
	The value is now ‘\the\count0’.
	\newline\\
	% dimen <0~255> -> fixed pont number
	\dimen0=0.001pt
	Float point value \the\dimen0\\
	\\
	%toks -> string type
	\newtoks\slbstr
	\slbstr={NICE}%
	This is \the\slbstr

	\count0=1 \count1=2
	\advance\count0 by \count1 %note that it actually use the "by" as operator
	count0 updated to \the\count0

\section{Macros}

\def\slbMacro{A replacement text}
Executing the macro "\slbMacro"
%arguments:
\def\slbArgMacro#1{replacement with argument=#1}
Invoking "\slbArgMacro{Hello!}"

\def\point(#1,#2){Point that take two args ->  #1 and #2}
\def\temp{(1,2)}
\expandafter\point\temp

\section{Scoping}
\def\x{1}

{
    \def\x{2}
    Value of x is now \x -> 2

    \global\def\y{3}
}

Value of x is now \x -> 1

Extract value from local scope \y ->3

\section{branching}

1=2? \ifnum 1 = 2 %
\else
No!
\fi

1=1? \ifnum 1 = 1
yes1
\else
\fi

\newcount\exactone %
\exactone=1
\advance\exactone by 1
1+1=2? \ifnum \exactone = 2
yes\fi


\end{document}