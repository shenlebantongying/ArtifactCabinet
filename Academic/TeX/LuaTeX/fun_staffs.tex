\documentclass[12pt,a4paper]{article}
\usepackage[default]{fontsetup}
\usepackage{microtype}

\usepackage{hyperref}

\usepackage{amsmath}
\usepackage{unicode-math}

\usepackage{tikz}
\usetikzlibrary{shapes.geometric}
\usetikzlibrary{shapes.misc}
\usetikzlibrary{positioning}
\usetikzlibrary{calc}
\usetikzlibrary{backgrounds}

\tikzset{
  every picture/.append
    style={
      background rectangle/.style={fill=yellow!10},
      show background rectangle}
}

\usepackage{listings}
\lstdefinestyle{mystyle}{
	numbers=left,
	basicstyle=\fontspec{Cascadia Mono},
	frame=single}

\lstset{style=mystyle}

\setlength\parindent{0pt}

\definecolor{racketBracket}{RGB}{0,178,178}
\definecolor{racketText}{RGB}{34,34,119}
\definecolor{racketComment}{RGB}{119,34,119}
\definecolor{racketString}{RGB}{34,119,34}

\lstdefinelanguage{racket}{
    morekeywords={},
    sensitive=false,
    morecomment=[l];,
    morestring=[b]",
    otherkeywords={(,),[,]}
}

\lstdefinestyle{rackeStyle}{
    keywordstyle=\color{racketBracket},
    basicstyle=\ttfamily\color{racketText},
    commentstyle=\color{racketComment},
    stringstyle=\color{racketString},
    numberstyle=\color{black},
    rulecolor=\color{black}
}

\begin{document}

\section{Fancy Fonts}

Embed font, \fbox{like {\fontspec{Ubuntu Mono}  Ubuntu} Mono}, in a sentence

This line uses {\fontspec{Cascadia Mono} Cascadia Mono.}

$\mathbb{P}[]$

\section{lua Scritping}

\directlua{myfun = require "fun"}

\directlua{
myfun.LaTeX();
tex.print("\\\\")
myfun.plus(3,4);
}

\section{Python listing}

\begin{lstlisting}
import os
import random
from itertools import groupby

import matplotlib.pyplot as plt

random.seed(os.urandom(128))

ball_move_Left = -1
ball_move_Right = 1


def oneBall(stages: int) -> int:
    return sum([random.choice([ball_move_Left, ball_move_Right]) for _ in range(stages)])


def galton(stages: int, n_balls: int):
    all_results = sorted([oneBall(stages) for _ in range(n_balls)])
    position = []
    counter = []
    for k, g in groupby(all_results):
        position.append(k)
        counter.append(len(list(g)))

    fig, ax = plt.subplots()
    ax.bar(position, counter)

    ax.set_xlim([-stages, stages])
    ax.set_xlabel("Position")
    ax.set_ylabel("Times")

    fig.show()


galton(100, 5000)
\end{lstlisting}

\section{Racket Listing}
\begin{lstlisting}[language=racket,style=rackeStyle]
; count the elements
(length (list "hop" "skip" "jump"))
3
\end{lstlisting}

\section{Draw staff with Tikz}

\subsection{Basis}

\begin{tikzpicture}
  \draw (-1,-1) -- (1,1);
  \draw (-1,1) -- (1,-1);
  \draw (0,0) circle [radius=0.5cm];
\end{tikzpicture}

\begin{tikzpicture}
  \draw (-1,-1) -- (1,1) [color=blue];
  \draw (-1,1) -- (1,-1) [color=green];
  \filldraw[black] (0,0) circle (2pt) node[anchor=west, color=red]{Center Dot};
\end{tikzpicture}

Looping, foreach

\begin{tikzpicture}
  \draw (0,0) \foreach \x in {1,2,...,5}{ -- (\x,0) -- (\x,\x)} ;
\end{tikzpicture}

\subsection{Basis 2}

Reuse coordinate

\begin{tikzpicture}
   \path (0,0) coordinate (LB);
   \path (3,3) coordinate (RT);
   \draw (LB) grid (RT);
   \draw (LB) -- (RT);
\end{tikzpicture}

Clipping, the red circle is clipped.

\begin{tikzpicture}
  \begin{scope}
    \draw[clip] (0,0) circle (1);
    \fill[blue] (1,0) circle (1);
  \end{scope}
  \fill[green,opacity=0.5] (-1,0) circle (1);
\end{tikzpicture}

Transforming, with coorinate reuse

\begin{tikzpicture}
  \path (0,0) coordinate (A);
  \path (1,3) coordinate (B);
  \draw [color=gray](A) rectangle (B);
  \begin{scope}[yshift=1]
    \draw [rotate=30] [purple] (A) rectangle (B);
  \end{scope}
\end{tikzpicture}

Transforming, without coorinate reuse

\begin{tikzpicture}
  \draw [color=gray](0,0) rectangle (1,3);
  \begin{scope}[yshift=1]
    \draw [rotate=30] [purple] (0,0) rectangle (1,3);
  \end{scope}
\end{tikzpicture}


\subsection{Tri-Circle}

\begin{center}
	\begin{tikzpicture}[->,scale=.7]
		\node (i) at (90:1cm) {$A$};
		\node (j) at (-30:1cm) {$B$};
		\node (k) at (210:1cm) {$C$};
		\draw (70:1cm) arc (70:-10:1cm) node[midway, right] {{\footnotesize 1}};
		\draw (-50:1cm) arc (-50:-130:1cm) node[midway, below] {{\footnotesize 2}};
		\draw (190:1cm) arc (190:110:1cm) node[midway, left] {{\footnotesize 3}};
	\end{tikzpicture}
\end{center}

\subsection{Diagrams}

\begin{tikzpicture}[
   node distance=1cm and 1cm,
   my/.style={
     rectangle,
     very thick,
     draw=blue!50!black!50,
   }]
  \node (one)[my]{Node 1};
  \node (two) [my, below=of one] {Node 2};
  \node (three) [my, above right=of two, yshift=-0.5cm] {Node 3};

  \path (one) edge[->] (two)
        (two) edge[->] (three);

\end{tikzpicture}


\section{PDF calculator}

Works in Firefox.\\

\begin{Form}
	\noindent%
	\TextField[name=a]{a:} \\ \\
	\TextField[name=b]{b:} \\ \\
	\TextField[name=c]{c:} \\ \\
	\noindent%
	$\sum = $ \TextField[name=AvgStat, calculate={
		event.value = (this.getField("a").value + this.getField("b").value + this.getField("c").value);
	}, readonly, value=0]{}
\end{Form}


\end{document}
