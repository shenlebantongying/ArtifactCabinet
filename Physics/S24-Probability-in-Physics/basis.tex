\documentclass[legalpaper]{article}

\usepackage{mathtools}
\usepackage[default]{fontsetup}

\usepackage{hyperref}

\usepackage{amsthm}
\usepackage{unicode-math}

\usepackage{tcolorbox}
\tcbuselibrary{theorems}

\NewTcbTheorem{define}{Definition}{}{th}
\NewTcbTheorem{theorem}{Theorem}{}{th}
\NewTcbTheorem{example}{Example}{}{th}

\setlength\parindent{0pt}
\AddToHook{cmd/section/before}{\clearpage}
\usepackage{nopageno}

% MY ===

\newcommand{\prob}{\mathbb{P}}
\newcommand{\samplespace}{\Omega}
\newcommand{\g}{$\rightarrow\ $}

\begin{document}

\section{Math Basis}

\begin{theorem}{Binomial theorem}{}
  \[
    (a+b)^n = \sum_{k=0}^n \binom{n}{k} a^{n-k}b^k
  \]
  where $\binom{n}{k} = \frac{n!}{k!(n-k)!}$.
\end{theorem}

\section{Conditional Probability}


\begin{define}{}{}
   The conditional probability of A given B is $\prob(A|B) = \frac{\prob(A\cap B)}{\prob(B)}$.
\end{define}

\begin{define}{}{}
  Two events are statistically independent if $\prob(A\cap B) = \prob(A)\prob(B)$
\end{define}

Equivilent defination \g $\prob(A|B) = \prob(A)$.

\begin{theorem}{}{}
  Bayes’ theorem \g $\prob(A|B) = \frac{\prob (B|A)\prob(A)}{\prob(B)}$.
\end{theorem}


\section{Random variables}

PMF \g Probability mass function

\subsection{Binomial}

\begin{define}{}{}
  Let X be a binomial random variable.

  PMF of X is
  \begin{align}
    P_x(k) = \binom{n}{k}p^k(1-p)^{n-k} &&k=0,1,...n.
  \end{align}
  where $p$ is the binomial paramater, and n is the total number of stats.

\end{define}

\end{document}