\documentclass[11pt,a4paper]{article}
\usepackage{geometry}
\geometry{margin=1in}
\usepackage{mathtools}
\usepackage[default]{fontsetup}

\usepackage{xcolor}
\definecolor{DodgerBlue4}{RGB}{16,78,139}
\usepackage[colorlinks=true,allcolors=DodgerBlue4]{hyperref}

\usepackage{cleveref}
\usepackage{amsthm}
\usepackage{unicode-math}

\usepackage{tcolorbox}
\tcbuselibrary{theorems}

\tcbset{
boxrule=1pt,
arc=1px,
parbox=false, %?
halign=left,halign upper=left,halign lower=left,
coltitle=black,
colbacktitle=white,
colback=white}

\NewTcbTheorem[number within=section, list inside=facts,crefname={fact}{Fact}]{fact}{Fact}{}{fact}
\NewTcbTheorem[number within=section, list inside=execs,crefname={exec}{Exercise}]{exec}{Exercise}{}{}

\setlength\parindent{0pt}
\setlength{\parskip}{1.02\baselineskip}
\numberwithin{equation}{section}

\usepackage{nopageno}
\usepackage{tocbibind}
\usepackage{setspace}

% MY ===

\newcommand{\prob}{\mathbb{P}}
\newcommand{\samplespace}{\Omega}
\newcommand{\g}{$\rightarrow\ $}

\DeclarePairedDelimiter\ceil{\lceil}{\rceil}
\DeclarePairedDelimiter\floor{\lfloor}{\rfloor}

\begin{document}
\tcblistof{facts}{List of facts}\clearpage
\section{Approximation}
\begin{fact}{Taylor Approximation}{}
  Let $f$ be a continuous function with infinite derivatives.
  Let $a\in \textbf{R}$ be a fixed constant.

  The Taylor approximation of $f$ at $x=a$ is

  \begin{align}
    f(x) &= f(a) + f'(a)(x-a) + \frac{f''(a)}{2!}(x-a^2)+...\\
         &= \sum^{\infty}_{n=0}\frac{f^{(n)}(a)}{n!}(x-a)^n
  \end{align}
  where $f^{(n)}$ denotes the nth-order derivative of $f$.
\end{fact}


\begin{fact}{Exponential series}{}
  Let $x$ be any real number. Then,
  \[
    e^x = \sum_{k=0}^{\infty}\frac{x^k}{k!}.
  \]
  \tcblower
  \begin{proof}
    Let $f(x) = e^x$.
    The nth-order derivative of $f(x)$ is $f^{(n)}=1$.
    The Taylor approximation around $x=0$ (or $a=0$) is simply the right hand side.
  \end{proof}
\end{fact}

\begin{exec}
   Show that $\lim_{n\rightarrow\infty}(1+\frac{1}{n})^n=e$.
   \tcblower
   TODO
\end{exec}

\section{Counting}
\begin{fact}{Binomial theorem}{binom}
  \[
  (a+b)^n = \sum_{k=0}^n \binom{n}{k} a^{n-k}b^k
  \]
  where $\binom{n}{k} = \frac{n!}{k!(n-k)!}$.
\end{fact}

\subsection{The Division or Pigeonhole Principles}

\begin{fact}{Division Principle}{}
  If $n$ objects are placed into $k$ boxes,
  then at lease one box contains $\ceil{\frac{n}{k}}$ or more objects,
  and at lease one box contains $\floor{\frac{n}{k}}$ or fewer objects.
\end{fact}

\begin{exec}
  Pick 6 integers between 0 and 9. Show that 2 of them must add up to 9.
  \tcblower
  All possible ways of adding up to 9 are known:
  $(0,9),(1,8),(2,7),(3,6),(4,5)$

  At lease one pair will contain both number when putting picked 6 numbers into those pairs, because at lease one pair will contains $\ceil{\frac{9}{6}}=2$ objects, thus there must be one box that add up to 9.
\end{exec}

\subsection{Combinatorial Proof}

\begin{fact}{Pascal’s identity}{}
  \[
  \binom{n+1}{k} = \binom{n}{k} + \binom{n}{k-1}
  \]
\end{fact}

Also

\[
{n \choose k} = {n \choose n-k}
\]

\begin{exec}
  Show that $\sum_{k=0}^{n}{n \choose k}^2 = {2n \choose n}.$
  \tcblower
  Divide $2n$ in the right side into 2 equal sized parts A and B, where $size(A)=size(B)=n$.

  To choose n objects from 2n, we choose $k$ from A and $n-k$ from B.
  Sum up all possible values of $k$, which is $[0,n]$.

  \[
  {2n \choose n} = \sum_{k=0}^{n}{n \choose k}{n \choose n-k} = \sum_{k=0}^{n}{n \choose k}^2
  \]

  where ${n \choose i} = {n \choose n-k}$.
\end{exec}

\begin{exec}
  Show that $\sum_{k=0}^{n} k(n+1-k)={n+1 \choose 3}$.
  \tcblower
  The right hand side can be interpreted as choosing 3 objects $\{j,k,l\}$ from $\{0,1,...,n,n+1\}$, where $0<=j<k<l<=n+1$.

  Object $j$ has $k$ choices while object $l$ has $n+1-k$ choices. Using the multiplication principle, for any given value of $k$, the total ways of picking $j$ and $l$ is $k(n+1-k)$. Summing them up, we get the left hand side.
  \qed

  Similarly, expression $1+2+3+...+n-1+n$ can be viewed as choosing two objects from both side of $k$ in set $\{0,1,2,3....n-1,n\}$ where $0<k<n, k \in \mathbb{Z}$, and thus $\sum_{i=1}^{n}i={n+1 \choose 2}$.
\end{exec}
\section{Probability basis}

\begin{fact}{Conditional probability}{}
  The probability of A given B is $\prob(A|B) = \frac{\prob(A\cap B)}{\prob(B)}$.
\end{fact}

\begin{fact}{Statistically independent}{}
  Two events are independent if $\prob(A\cap B) = \prob(A)\prob(B)$
\end{fact}

Equivalent definition \g $\prob(A|B) = \prob(A)$.

\begin{fact}{Bayes’ theorem}{}
  \g $\prob(A|B) = \frac{\prob (B|A)\prob(A)}{\prob(B)}$.
\end{fact}

\section{Random variables}

PMF \g Probability mass function

PDF \g Probability density function

CDF \g Cumulative distribution functions

\subsection{Binomial}

\begin{fact}{Binomial random variable}{}
  Let X be a Binomial random variable.

  PMF of X is
  \begin{align}
    f_X(k) = \binom{n}{k}p^k(1-p)^{n-k} &&k=0,1,...
  \end{align}
  where $p$ is the binomial parameter, and n is the total number of stats.

\end{fact}

\subsection{Poisson}
A special case of Binomial distribution, where $p\rightarrow0,n\rightarrow\infty$.

\begin{fact}{Poisson random variable}{}
    Let X be a Poisson random variable.
    X is drawn from a Poisson distribution with a parameter $\mu$.

    PMF of X is
    \begin{align}
      f_X(k) =  \frac{\mu^k}{k!}e^{-\mu} && k = 0,1,...
    \end{align}
\end{fact}

\end{document}