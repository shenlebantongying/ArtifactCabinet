\section{Random variables}

\subsection{Basis of random variables}

\begin{fact}{Random variable definition}{}
  A random variable $X$ is a function $X:\samplespace\rightarrow\mathbb{R}$ that maps an outcome $o\in\samplespace$ to a number $X(o)$ on the real line.

  Note that $\Omega$ is the sample space.
\end{fact}

\begin{fact}{PMF \g Probability mass function}{}
    The PMF of a discrete random variable $X$ is the function $p_X$ given by $p_X(x) = \mathbb{P}(X=x)$.

    Note that $X=x$ denote an event, consisting of all outcomes to which $X$ assigns the number $x$.
\end{fact}

\begin{fact}{PDF \g Probability density function}{}
  Let $X$ be a continuous random variable.

  The PDF of X is a function $f_X\mathbin{:}\Omega\rightarrow\mathbb{R}$ that, when integrated
  over an interval $[a,v]$, yields the probability of obtaining $a\le X \le b$:
  \begin{equation*}
    \mathbb{P}[a\le X \le b] = \int_{a}^{v}f(x)dx.
  \end{equation*}
  Basic properties:
  \begin{itemize}
    \item Unity: $\int_\samplespace f(x)dx=1$ (Integration over the entire sample space of PDF yields 1.)
  \end{itemize}
\end{fact}

\begin{fact}{CDF \g Cumulative distribution functions}{}
    Let $X$ be a continuous random variable with sample space $\samplespace=\mathbb{R}$.
    The CDF of $X$ is
    \begin{equation*}
      F_X(x)=\mathbb{P}(X\le x)=\int_{-\infty}^{x}f_X(t)dt.
    \end{equation*}
\end{fact}

\begin{fact}{Expectation definition}{}
    The \textbf{expectation} of a continuous random variable X is
    \begin{equation*}
        \mathbb{E}[X] = \int_{\samplespace}xf_X(x)\,dx.
    \end{equation*}
    The \textbf{expectation} of a discrete random variable X is
    \begin{equation*}
        \mathbb{E}[X]=\sum_{x\in \samplespace}x f_X(x).
    \end{equation*}
\end{fact}

\begin{fact}{Variance definition}{}

    The \textbf{variance} of a continuous random variable X is
    \begin{equation*}
        \Var[X] = \mathbb{E}[(X-\mu)^2]=\int_{\samplespace}(x-\mu)^2f_X(x)\,dx
    \end{equation*}
    where $\mu=\mathbb{E}[X]$ is the expectation of $X$.
\end{fact}

Note that if a function $g$ is applied to random variable $X$, the expectation can be found via
\begin{equation*}
  \mathbb{E}[g(X)]=\int_{\samplespace}g(x)f_X(x)\,dx.
\end{equation*}

\subsection{Bernoulli}

\begin{fact}{Bernoulli random variable}{}
    The expression
    \begin{equation*}
        X \sim Bernoulli(p)
    \end{equation*}
    means $X$ is drawn form a Bernoulli distribution with parameter $p$
    , which is controls the probability of obtaining 1.

    The PMF of $X$ is
    \begin{align*}
        p_X(0) &= 1-p\\
        p_X(1) &= p
    \end{align*}
    where $0<p<1$.
\end{fact}

\subsection{Binomial}

\begin{fact}{Binomial random variable}{}
  For $X \sim Binomial(n,p)$, the PMF of X is
  \begin{align*}
    f_X(k) = \binom{n}{k}p^k(1-p)^{n-k} &&k=0,1,...
  \end{align*}
  where $p$ is the binomial parameter, and n is the total number of stats.
\end{fact}

\subsection{Poisson}
A special case of Binomial distribution, where $p\rightarrow0,n\rightarrow\infty$.

\begin{fact}{Poisson random variable}{}
  Random variable $X$ is drawn from a Poisson distribution with a parameter $\lambda$.
  \begin{equation*}
      X \sim \text{Poisson}(\lambda)
  \end{equation*}

  PMF of X is
  \begin{align*}
    f_X(k) =  \frac{\lambda^k}{k!}e^{-\lambda} && k = 0,1,...
  \end{align*}
  Variable $k$ is the number of events that occurs in a range of time or a region of space (or whatever).

  The parameter $\lambda$ determines the rate of the arrival or occurrence.
\end{fact}

\begin{exec}
  Show that if $X\sim \text{Poisson}(\lambda)$, then $\mathbb{E}[X]=\lambda$.
  \tcblower
  \begin{align*}
    \mathbb{E}[X]
    &= \sum_{k=0}^{\infty} k\frac{\lambda^k}{k!}e^{-\lambda}\\
    &= \sum_{k=1}^{\infty}\frac{\lambda^k}{(k-1)!}e^{-\lambda} = \lambda e^{-\lambda}\sum_{k=1}^{\infty}\frac{\lambda^{k-1}}{(k-1)!}\\
    &= \lambda e^{-\lambda}\sum_{r=0}^{\infty}\frac{\lambda^{r}}{r!} = \lambda e^{-\lambda}e^{\lambda}\\
    &= \lambda.
  \end{align*}
  Note that $e^x=\sum_{n=0}^{\infty}\frac{x^n}{n!}$ (The Maclaurin series of $e^x$).
\end{exec}

\subsection{Gaussian}

\begin{fact}{Gaussian random variable}{}
  Let $X$ be a Gaussian (Normal) random variable.
  \begin{equation*}
      X \sim \mathcal{N}(\mu,\sigma^2)
  \end{equation*}
  Its PMF of is
  \begin{equation*}
    f(x)=\frac{1}{\sqrt{2\pi\sigma^2}}exp\left[-\frac{(x-\mu)^2}{2\sigma^2}\right],
  \end{equation*}
  where $(\mu\rightarrow mean,\sigma^2\rightarrow variance)$ are parameters of the distribution.

  It can also be rewritten by letting $b=\frac{1}{2\sigma^2}$, and thus
  \begin{equation*}
    f(x)=\sqrt{\frac{b}{\pi}}e^{-b(x-\mu)^2}.
  \end{equation*}

  Its expectation $\mathbb{E}[X]=\mu$ and variance $\Var[X]=\sigma^2$.

  \textbf{Standard Gaussian} random variable is a Gaussian with $\mu=0$ and $\sigma^2=1$. Its PDF is
  \begin{equation*}
    f(x)=\frac{1}{\sqrt{2\pi}}e^{-\frac{x^2}{2}}
  \end{equation*}
\end{fact}

\begin{exec}
  Using multivariate Standard Gaussian distribution to obtain random points on a sphere.
  \tcblower
  Suppose that $x,y,z$ are 3 independent random variables that follows Standard Gaussian distribution.
  The probability of $x,y,z$ are certain values $(x',y',z')$ is
  \begin{align*}
    f(x')f(y')f(z')=\frac{1}{\sqrt{2\pi}}e^{-\frac{x'^2+y'^2+z'^2}{2}}.
  \end{align*}
  Normalizing $(x',y',z')$ by multiplying each with $r'=\frac{1}{\sqrt{x'^2+y'^2+z'^2}}$ will yield a point $(\frac{x'}{r'},\frac{y'}{r'},\frac{z'}{r'})$ at a unit sphere.
\end{exec}

\begin{fact}{CDF of standard Gaussian and Error function}{}
  For standard Gaussian, the CDF is defined as
  \begin{equation*}
    \Phi(x)=\int_{-\infty}^{x}f(t)dt=\frac{1}{\sqrt{2\pi}}\int_{-\infty}^{x} e^{-\frac{t^2}{2}}\,dt.
  \end{equation*}
  The \textbf{error function} is defined as
  \begin{equation*}
    \Erf (x)=\frac{2}{\sqrt{\pi}}\int_{0}^{x}e^{-t^2}\,dt.
  \end{equation*}
\end{fact}

\begin{exec}
  Show that $\Erf(x)=2\Phi(x\sqrt{2})-1$
  \tcblower
  \TODO
  \begin{align*}
    2\Phi(x\sqrt{2})-1
    &= 2\frac{1}{\sqrt{2\pi}}\int_{-\infty}^{x\sqrt{2}}e^{-\frac{t^2}{2}\,dt}-1\\
  \end{align*}
  \TODO hint: polar coordinate and E
  \begin{equation*}
    \int e^{-x^2}\,dt = \int_{0}^{x} e^{t^2}\,dx = \frac{\sqrt{\pi}}{2}\Erf(x)
  \end{equation*}
\end{exec}

\begin{fact}{CDF of arbitrary Gaussian}{}
  Let $X\sim \mathcal{N}(\mu, \sigma^2)$
  \begin{equation*}
    F_X(x)=\Phi(\frac{x-\mu}{\sigma}) = \frac{1}{\sqrt{2\pi}}\int_{-\infty}^{\frac{x-\mu}{\sigma}} e^{-\frac{t^2}{2}}\,dt.
  \end{equation*}
\end{fact}

\begin{fact}{Convolution}{}
  The convolution of two function $f$ and $g$ for a value $x$ over infinite range $t$ is given by
  \begin{equation*}
    (f*g)[x] = \int_{-\infty}^{\infty}f(t)g(x-t)\,dt.
  \end{equation*}
\end{fact}

For two random variables $X$ and $Y$, the PDF of $X+Y$ is equivalent to convolving the PDFs of $X$ and $Y$.

\subsection{Multiple independent random variables}

\begin{fact}{Sample mean \g  Expectation}{sampleMean}
  Let $X_1,X_2,...,X_n$ be random sample of size $n$ from a distribution with mean $\mu$ and variance $\sigma^2$.

  Sample mean is a linear combination of independent random variables
  \begin{equation*}
    \mean{X}=\frac{X_1+X_2+...+X_n}{n}.
  \end{equation*}

  The expectation is
  \begin{equation*}
    \mathbb{E}[\mean{X}]=\mathbb{E}[\frac{X_1+X_2+...+X_n}{n}]=\frac{1}{n}[\mu+\mu+...+\mu]=\mu.
  \end{equation*}
\end{fact}

\begin{exec}
  Suppose $X_1,X_2,...,X_n$ are $n$ independent random variables with means $\mu_1,\mu_2,...\mu_n$ and variances $\sigma_1^2,\sigma_2^2,...,\sigma_n^2$.

  Show that the variance of the linear combination $Y=\sum_{i=1}^{n}a_iX_i$ is $\sigma_Y^2=\sum_{i=1}^{n}a_i^2\sigma_i^2$ where $a_1,a_2,...,a_n$ are real constants.
  \tcblower
  By definition of the variance.
  \begin{equation*}
    \sigma_Y^2=\Var(Y)=\mathbb{E}[(Y-\mu_Y)^2]
  \end{equation*}
  Then, substitute the value and the mean of of $Y$.
  \begin{align*}
    \sigma_Y^2
    &=\mathbb{E}\left[\left(\sum_{i=1}^{n}a_iX_i-\sum_{i=1}^{n}a_i\mu_i\right)^2\right]\\
    &=\mathbb{E}\left[\left(\sum_{i=1}^{n}a_i(X_i-\mu_i)\right)^2\right]\\
    &=\mathbb{E}\left[\left(\sum_{i=1}^{n}a_i(X_i-\mu_i)\right) \left(\sum_{i=j}^{n}a_j(X_j-\mu_j)\right)\right]\\
    &=\mathbb{E}\left[\sum_{i=1}^{n}\sum_{i=j}^{n}a_ia_j(X_i-\mu_i)(X_j-\mu_j) \right]\\
    &=\sum_{i=1}^{n}\sum_{i=j}^{n}a_ia_j\mathbb{E}\left[(X_i-\mu_i)(X_j-\mu_j) \right]
  \end{align*}
  Expand this summation. Note that when $i=j$, the term's expectation is the variance of $X_i$, and when $i\neq j$, the term's expectation is the covariance between independent random variable $X_i$ and $X_j$ which is zero.
  \begin{align*}
    \sigma_Y^2
    =&\mathcolor{blue}{a_1a_1\mathbb{E}[(x_1-\mu_1)(x_1-\mu_1)]}+
      \mathcolor{red}{a_1a_2\mathbb{E}[(x_1-\mu_1)(x_2-\mu_2)]}+
      \mathcolor{red}{a_1a_3\mathbb{E}[(x_1-\mu_1)(x_3-\mu_3)]}+ ... +\\
     &\mathcolor{red}{a_2a_1\mathbb{E}[(x_2-\mu_2)(x_1-\mu_1)]}+
      \mathcolor{blue}{a_2a_2\mathbb{E}[(x_2-\mu_2)(x_2-\mu_2)]}+
      \mathcolor{red}{a_2a_2\mathbb{E}[(x_2-\mu_2)(x_3-\mu_3)]}+ ... +\\
     &\mathcolor{red}{a_na_1\mathbb{E}[(x_n-\mu_n)(x_1-\mu_1)]}+
      \mathcolor{red}{a_na_2\mathbb{E}[(x_n-\mu_n)(x_2-\mu_2)]}+...+
      \mathcolor{blue}{a_na_n\mathbb{E}[(x_n-\mu_n)(x_n-\mu_n)]}\\
    =&\mathcolor{blue}{a_1^2\mathbb{E}[(X_1-\mu_1)^2]+a_2^2\mathbb{E}[(X_2-\mu_2)^2]+...+a_n^2\mathbb{E}[(X_n-\mu_n)^2]}\\
    =&a_1^2\mu_1^2+a_2^2\mu_2^2+...+a_n^2\mu_n^2\\
    =&\sum_{i=1}^{n}a_i^2\sigma_i^2
  \end{align*}
  \TODO read covariance.
\end{exec}

\begin{fact}{The variance of sample mean}{}
  If every $X_i$ is the same distribution and $a_i=1$ in the above exercise, the variance of the sample mean is
  \begin{equation*}
    \Var[\mean{X}]=\frac{\sigma^2}{n}.
  \end{equation*}
\end{fact}

