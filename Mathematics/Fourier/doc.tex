\setuppagenumbering
  [location=]

\setupfloat
  [figure]
  [default={here}]
\setupexternalfigure
  [maxwidth=\textwidth]

\setupmathalignment
  [n=2,align={1:right,2:left}]

\setupmathematics
  [integral=nolimits]

\setupinteraction[state=start]
\placebookmarks[section,subsection]

\setupformulae
  [location=left]

\starttext
\title{Joseph Fourier}

\section{Backgrounds}

\subsection{Fourier Series}

The \emph{Fourier Series} of \m{f(x)}:

\startformula
  f(x) \sim \frac{a_0}{2} + \sum^\infty_{k=1} a_k \cos(\frac{k \pi x}{l}) + b_k \sin(\frac{k \pi x}{l})
\stopformula

The period of function \m{f(x)} is \m{T=2l}.

Note the \m{\sim} is used instead of \m{=} because we don't know if the sum would converges or diverges.

This function can be simplified:

\startformula
  F(t) = f(tl/\pi) \sim \frac{a_0}{2} + \sum_{k=1}^\infty(a_k \cos(kt) + b_k \sin(kt))
\stopformula

This has a fundamental period of \m{2\pi}.

\startformula\startalign
  \NC a_k \NC= 1/\pi \int^\pi_\pi F(t) \cos(kt) dt, k=0,1,2,3 ... \NR
  \NC b_k \NC= 1/\pi \int^\pi_\pi F(t) \sin(kt) dt, k=0,1,2,3 ...
\stopalign\stopformula

\subsubsection{Example \m{f(x)=-t/2} in \m{[-pi,pi]}}

Let's consider function \m{f(t)=-t/2} on interval \m{(-\pi,\pi)}. Expanding it periodically with period \m{2\pi}.

\placefigure{f(t)}{\externalfigure[fig1_fx_theory.pdf]}

\startformula\startalign
  \NC a_0 \NC= 1/\pi \int^\pi_{-\pi} -t/2 dt = 0 \NR
  \NC a_k \NC= 1/\pi \int^\pi_{-\pi} \cos(kt) dt = -\frac{1}{2 \pi k^2}(tk \sin(kt) + \cos(kt))|^{t=\pi}_{t=-\pi} = 0 \NR
  \NC b_k \NC= 1/\pi \int^\pi_{-\pi} -\frac{-t}{2} \sin(kt) dt \NR
  \NC     \NC= \frac{1}{2 \pi k^2}(kt \cos(kt) - \sin(kt))|^{t=pi}_{t=-pi} \NR
  \NC     \NC= \frac{(-1)^k}{k}
\stopalign\stopformula


Then we can obtain
\startformula
f(t) = -t/2 = \sum^\infty_{k=1} \frac{(-1)^k}{k} \sin(kt), -\pi < t < \pi
\stopformula

\placefigure{f(t) in Fourier Series}{\externalfigure[fig1_fx_fourier.pdf]}

\stoptext
