\documentclass[12pt,letterpaper,fleqn]{article}
\usepackage{mathtools}
\mathtoolsset{showonlyrefs}
\usepackage{amssymb}
\usepackage{fontsetup}
\usepackage{nopageno}
\usepackage[paperheight=12in]{geometry}
\setlength\parindent{0pt}
\setcounter{secnumdepth}{0}
\begin{document}
\section{Variance formula for grouped data}
If a variable $X$ that has $k$ possible values
\begin{equation}
    X_1,X_2,X_3...,X_k
\end{equation}
occurring with respective frequencies
\begin{equation}
    f_1,f_2,f_3,...,f_k.
\end{equation}
The total number of samples is
\begin{equation}
    n = \sum_{i=1}^{k} f_i
\end{equation}
The mean of these values is
\begin{equation}
    \bar{X} = \frac{\sum_{i=1}^{k} f_i X_i}{n} = \frac{\sum fX}{n}
\end{equation}

The sample variance for a sample drawn from random variable Y is
\begin{equation}
    s^2 = \sum_{i=1}^{n}(Y_i - \bar{Y})^2/(n-1).
\end{equation}

Since we already knows frequencies, the sample variance for $X$ is thus
\begin{align}
    s^2
    &= \frac{\sum_{i=1}^{k}f_i (X_i-\bar{X})^2}{n-1} \\
    &= \frac{\sum_{i=1}^{k} f_i (X_i^2 - 2 X_i\bar{X}+\bar{X}^2)}{n-1} \\
    &= \frac{\sum f X^2 - \sum 2 f X\bar{X} + \bar{X}^2 \sum f }{n-1} \\
    &= \frac{\sum f X^2 - 2 \bar{X} \sum f X + \bar{X}^2 \sum f }{n-1}.
\end{align}
Note that $\sum f X = n \bar{X}$.
\begin{align}
    s^2
    &= \frac{\sum f X^2 - 2n \bar{X} + n\bar{X}^2}{n-1} \\
    &= \frac{\sum f X^2 - n \bar{X}^2}{n-1} \\
    &= \frac{\sum f X^2 - n (\frac{\sum fX}{n})^2}{n-1} \\
    &= \frac{1}{n-1} \left[ \sum f X^2 - \frac{(\sum fX)^2}{n} \right].
\end{align}
\end{document}